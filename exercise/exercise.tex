\documentclass{exam}
\usepackage[utf8x]{inputenc}
\usepackage[ngerman]{babel}
\usepackage[german,ruled,vlined,linesnumbered,commentsnumbered,algoruled]{algorithm2e}
\usepackage{amsmath}
\usepackage{amsfonts}
\usepackage{amssymb}
\usepackage{amsthm}
\usepackage{marvosym}
\usepackage{tikz}
\usepackage[pdftex,
            pdfauthor={Fabian Feitsch},
            pdfsubject={Irrgarten-Aufgabe},
            pdfproducer={Latex},
            pdfcreator={pdflatex}]{hyperref}
\usepackage{csquotes}
\usepackage{censor}
\usepackage{listings}
\usepackage{parskip}
\SetKw{To}{to}
\SetKw{KwInt}{int}
\SetKw{KwTrue}{true}
\SetKw{KwFalse}{false}
\SetKw{KwOr}{or}

\DontPrintSemicolon
\newcommand{\A}{\sffamily{A}\normalfont}
\newcommand{\ident}[1]{{\normalfont\sffamily{#1}}}
\renewcommand\familydefault{\sfdefault}

\pagestyle{headandfoot}
\runningheadrule
\header{\today}{Irrgarten-Aufgabe für Java-Vorlesung}{}
\footer{}{}{Seite \thepage}
\DeclareUnicodeCharacter{9608}{\censor{B}}

\title{\vspace{-1cm}\sffamily{\textbf{Irrgarten-Aufgabe für Java-Vorlesung}}}
\date{}

 
\begin{document}
%\printanswers
\maketitle

\qformat{\textbf{\sffamily{Aufgabe \thequestion: \thequestiontitle}} \hfill }
\renewcommand{\solutiontitle}{\noindent\textbf{Lösung: }}
\renewcommand{\questionshook}{%
\setlength{\leftmargin}{0pt}%
\setlength{\labelwidth}{-\labelsep}%
}
\noindent{}In dieser Aufgabe sollen Sie Schritt für Schritt ein Java-Programm entwickeln, welches den kürzesten Weg aus einem Irrgarten findet. Das Java-Programm soll später mit folgendem Befehl gestartet werden können:


\begin{lstlisting}
java de.yourname.maze.cmd.Solver /home/user/maze.txt 0 3 9 6
\end{lstlisting}

Das erste Argument ist der Dateipfad zu einer Datei, in der der Irrgarten definiert wird. Das zweite Argument gibt an, in welcher Zeile des Irrgartens gestartet werden soll. Das dritte Argument gibt die Start-Spalte an. Das vierte und fünfte Argument definieren den Ausgang des Irrgartens analog zur Start-Definition. Die Ausgabe des Programms soll den Irrgarten zeigen, sowie den kürzesten Weg durch den Irrgarten:

\begin{center}
\begin{minipage}{2cm}
\begin{verbatim}
███*██████
█***██   █
█*█    █ █
█*██████ █
█*█      █
█*█ █ ████
█*█ █    █
█*█ ████ █
█******█ █
██████*███
\end{verbatim}
\end{minipage}
\end{center}
\paragraph{Hinweise:} Achten Sie auf die Zugriffsmodifizierer der Klassen und Methoden. Sie dürfen keine öffentlichen Methoden außer denen programmieren, die explizit in derAufgabe verlangt werden. Sofern nicht anders angegeben, sind die Methoden nicht statisch. Sie dürfen jedoch beliebige \verb|private|-Methoden implementieren, wenn diese Ihnen helfen. Achten Sie auch darauf, für Ihre Klassen das in den jeweiligen Aufgaben verlangte Paket auszuwählen.

\begin{questions}
\titledquestion{Importieren des Irrgartens}\label{exec:importer}
Schreiben Sie eine Klasse \verb|Importer|, die im Paket \verb|de.yourname.maze| liegt. Diese Klasse soll eine öffentliche Methode \verb|readFromScanner| besitzen, die ein zweidimensionales \verb|char|-Feld zurückgibt (\verb|char[][]|). Diese Methode nimmt ein Object vom Typ \verb|Scanner| entgegen und soll vom Scanner Eingaben des folgenden Formates einlesen:

\begin{lstlisting}
10
Lorem ipsum dolor sit amet,
consectetur adipiscing elit,
sed do eiusmod tempor incididunt
ut labore et dolore magna aliqua.
Mauris ultrices eros in cursus
turpis massa. Tristique
sollicitudin nibh sit amet commodo.
At imperdiet dui accumsan sit amet.
Fringilla urna porttitor rhoncus dolor purus.
Posuere morbi leo urna molestie at
\end{lstlisting}

In der ersten Zeile steht stets eine Zahl, die angibt, wie viele Zeilen noch kommen. Die restlichen Zeilen stellen den eigentlichen Inhalt dar und sollen in ein \verb|char[][]|-Feld gepackt werden. Zum Beispiel soll die Methode mit obiger Eingabe ein \verb|char[][]| zurückgeben, sodass \verb|char[2][0] = 's'| und \verb|char[2][1] = 'e'|. Sie dürfen davon ausgehen, dass in der ersten Zeile immer eine Zahl steht und diese die richtige Anzahl an Zeilen angibt.

\paragraph{Tipp:} Lassen sich sich vom Scanner eine komplette Zeile als String \verb|line| auslesen und wandeln Sie diese mittels der Methode \verb|line.toCharArray()| in ein Feld vom Typ \verb|char[]| um.

\titledquestion{PrettyPrinter}

Schreiben Sie eine Klasse \verb|PrettyPrinter| im Paket \verb|de.yourname.maze|. Die Klasse soll ein privates, finales Attribut vom Typ \verb|char[][]| besitzen, sowie einen Konstruktor mit einem Parameter, der das private Attribut setzt. Weiterhin soll die Klasse eine Methode \verb|printPretty| besitzen, die einen \verb|PrintStream| entgegen nimmt und das einzige Klassenattribut vom Typ \verb|char[][]| so ausgibt wie die Eingabe aus Aufgabe~\ref{exec:importer} aussah, aber ohne die Zahl in der ersten Zeile.

\paragraph{Tipp:} Mittels \verb|new String(char[])| können Sie ein \verb|char[]| in einen String umwandeln. Sie benötigen den Tipp nicht unbedingt, aber er macht den Code kürzer.

\titledquestion{Die Domain-Klassen}
Im folgenden werden die Klassen definiert, die den Irrgarten später darstellen sollen. Es handelt sich um die drei Klassen \verb|Cell|, \verb|CellList| und \verb|CellListItem|. Legen Sie alle drei Klassen im Paket \verb|de.yourname.maze.domain| an.
\begin{parts}
  \part
  Wir definieren den Begriff \emph{Zelle} als ein freies Feld in unserem Irrgarten. Eine Zelle besitzt vier Eigenschaften: \verb|row|, \verb|col|, \verb|neighbours| und \verb|predecessor|. Die ersten beiden Attribute beschreiben den Ort, an dem sich die Zelle im Irrgarten befindet. Welchen Datentyp schlagen Sie vor, um \verb|row| und \verb|col| zu verwalten?
  \part
  Erstellen Sie eine Klasse \verb|Cell| mit den Attributen \verb|row| und \verb|col| sowie einen Konstruktor, mit dem beide Werte gesetzt werden können. Schreiben Sie außerdem Getter für beide Attribute. Legen Sie eine \verb|toString()|-Methode an, die beide Werte schön formatiert ausgibt.
  \part
  Das Attribut \verb|neighbours| sei vom Typ \verb|CellList| und das Attribut \verb|predecessor| sei vom Typ \verb|Cell|.
\end{parts}
\titledquestion{Induktionsbeweis eines Algorithmus}
Betrachten Sie den in Aufgabe \ref{quest:schleifeninvariante} vorgestellten Algorithmus zur Berechnung der Fakultät.
\begin{parts}
  \part Geben Sie einen Algorithmus in Pseudocode an, der die Fakultät rekursiv berechnet.
  \begin{solution}
  
    \begin{algorithm}[H]
      \If{$k = 0$}{
        \Return $1$
      }
      \Return $k$ $\cdot$ \ident{fakultätRek($k-1$)}
      \caption{\ident{int fakultätRek}(\textbf{int} $k$)}
    \end{algorithm}
  \end{solution}
  \part Beweisen Sie mittels vollständiger Induktion die Korrektheit Ihrer rekursiven Variante.
  \begin{solution}
    Wir beweisen die Korrektheit mittels vollständiger Induktion über $k$:
    \begin{description}
      \item[Induktionsanfang] Für $k = 0$ ist $k! = 1$. Das gibt genau die \textbf{if}-Abfrage in Zeile 1 zurück.
      \item[Induktionsschritt] Sei der Algorithmus korrekt für ein beliebiges $k-1$. Wir zeigen, dass der Algorithmus dann auch für $k$ korrekt ist. Wir wissen, dass $k! = k \cdot (k-1)!$. Genau dies führt die Algorithmus in Zeile 3 durch. Da der Algorithmus für $k-1$ korrekt ist, ist auch die Berechnung für $k! = k \cdot (k-1)!$ korrekt.
    \end{description}
  \end{solution}
\end{parts}
\ifprintanswers
\else
\newpage
\fi
\titledquestion{Vollständige Induktion}
Zeigen Sie die folgenden Aussagen mittels vollständiger Induktion.
\begin{parts}
  \part Für jede natürliche Zahl $n$ ist 3 ein Teiler von $n^3-n$.
  \begin{solution}
    Wir beweisen die Aussage mit vollständiger Induktion über $n$:
    \begin{description}
      \item[Induktionsanfang] Sei $n = 1$. Dann ist $n^3=1$ und $n^3-n = 0$. Die Zahl 3 ist tatsächlich ein Teiler von 0.
      \item[Induktionsschritt] Sei die Aussage richtig für beliebiges, festes $n$. Wir zeigen, dass sie auch für $n+1$ richtig ist.
      \begin{equation*}
        \begin{split}
          (n+1)^3 - (n+1) &= n^3 + n^2 + 2n^2 + 2n + n + 1 - (n+1) \\
          &=(n^3 - n) + 3n^2+3n\\
          &=(n^3 - n) + 3(n^2+n)
        \end{split}
      \end{equation*}
      Laut Induktionsannahme ist $n^3-n$ durch 3 teilbar. Nun addieren wir zu einer durch 3 teilbaren Zahl ein Vielfaches von 3. Folglich ist die Summe ebenfalls durch 3 teilbar.
    \end{description}
  \end{solution}
  \part Die Fibonacci-Folge ist eine rekursiv definierte Zahlenfolge. Dabei ist $F(0) = 0$ und $F(1) = 1$. Die $n$-te Fibonacci-Zahl für ein $n > 1$ ist dann $F(n-1) + F(n-2)$. Die Berechnungsvorschrift dauert für große $n$ jedoch sehr lange. Mit der Formel von Moivre-Binet kann die $n$-te Fibonacci-Zahl direkt ausgerechnet werden. Beweisen Sie die Richtigkeit der Formel:
  \begin{equation*}
    F(n) = \frac{(\frac{1+\sqrt{5}}{2})^n - (\frac{1-\sqrt{5}}{2})^n}{\sqrt{5}}
  \end{equation*}
  \begin{solution}
    Sei $a= (1+\sqrt{5})/2$ und $b= (1-\sqrt{5})/2$. Wir zeigen die Korrektheit der Aussage durch eine Induktion über $n$.
    \begin{description}
      \item[Induktionsanfang] Sei $n' = 2$. Dann ist $F(n) = F(0) + F(1) = 1$. Setzen wir $n'=2$ direkt in die obige Gleichung ein, erhalten wir ebenfalls $1$. Da die Fibonacci-Zahlen auf den jeweils zwei vorherigen Folgengliedern aufbauen, müssen wir auch $n'=3$ testen: $F(n) = F(2) + F(1) = 3$, was mit dem Ergebnis der direkten Gleichung übereinstimmt.
      \item[Induktionschritt] Sei die Aussage richtig für $n-1$ und $n-2$. Wir zeigen, dass die Aussage dann auch für $n$ richtig ist:
      \begin{equation*}
        \begin{split}
          F(n) = F(n-1) + F(n-2) &\underbrace{=}_{\text{IA}} \frac{a^{n-1} - b^{n-1}}{\sqrt{5}} + \frac{a^{n-2} - b^{n-2}}{\sqrt{5}}\\
          &= \frac{a^{n-1}- b^{n-1} + a^{n-2} - b^{n-2}}{\sqrt{5}}\\
          &= \frac{a^{n-1}(1+\frac{1}{a})- b^{n-1}(1+\frac{1}{b})}{\sqrt{5}}
        \end{split}
      \end{equation*}
      Es wäre schön, wenn $1+1/a = a$. Also überprüfen wir das:
      \[ 1+\frac{1}{a} = a \Rightarrow a+1 = a^2 \Rightarrow a^2-a-1 = 0 \Rightarrow a = \frac{1\pm\sqrt{1-4\cdot(-1)}}{2} = \frac{1\pm\sqrt{5}}{2}\]
      Wir setzen $a = 1+1/a$ oben ein und erhalten das gewünschte Ergebnis:
      \[ \frac{a^{n-1}a - b^{n-1}b}{\sqrt{5}} = \frac{a^n - b^n}{\sqrt{5}} = \frac{(\frac{1+\sqrt{5}}{2})^n - (\frac{1-\sqrt{5}}{2})^n}{\sqrt{5}}\]
    \end{description}
  \end{solution}
  \part Auf einem quadratischen Schachbrett mit einer Seitenlänge von mehr als drei Feldern kann der Springer jedes Feld von jedem anderen Feld erreichen. Dafür hat er beliebig viele Züge zur Verfügung.
  \begin{solution}
    Wir zeigen die Aussage durch Induktion über die Seitenlänge $n$ in Feldern:
    \begin{description}
      \item[Induktionsanfang] Für $n=3$ finden wir durch Versuchen heraus, dass die Aussage stimmt.
      \item[Induktionsschritt] Sei die Aussage für $n-1$ bewiesen. Wir zeigen die Richtigkeit der Aussage für ein beliebiges $n > 3$. Dank der Induktionsannahme wissen wir, dass sich der Springer im Teilbrett $n-1 \times n-1$ überallhin bewegen kann. Betrachten wir das Brett $n \times n$. Alle Randfelder sind durch nur einen \enquote{Springersprung} von einem Mittelfeld erreichbar. Jedes Mittelfeld ist aber wegen der Induktionsannahme ebenfalls erreichbar. Deshalb kann der Springer auch im $n \times n$ Schachbrett mit beliebig vielen Sprüngen auf alle Felder springen.
    \end{description}
  \end{solution}
\end{parts}
\titledquestion{Schleifeninvariante}\label{quest:schleifeninvariante}
Gegeben sei der folgende Algorithmus, der die Fakultät einer Zahl $k$ berechnet.
 \begin{algorithm}
  $f=j=k$\;
  \While{$j>1$}{
    $j=j-1$\;
    $f=f\cdot j$\;
  }
  \Return $f$\;
 	\caption{\ident{int fakultät}(\ident{int} $k$)}
\end{algorithm}
\begin{parts}
  \part\label{quest:invariante} Geben Sie eine geeignete Invariante an.
  \begin{solution}
    Vor der $i$.\ Iteration der \textbf{while}-Schleife hat $f$ den Wert $k!/(k-i)!$.
  \end{solution}
  \part Zeigen Sie mit Hilfe der in \ref{quest:invariante}) aufgestellten Invariante die Korrektheit des Algorithmus.
  \begin{solution}
    \begin{description}
      \item[Initialisierung] Vor der ersten $(i=1)$ Iteration hat $f$ den Wert $k!/(k-i)! = k!/(k-1)! = k$, was nach der Zuweisung in Zeile 1 korrekt ist.
      \item[Aufrechterhaltung] Sei die Invariante korrekt vor der $i$.\ Iteration. Dann wird innerhalb der Schleife der Wert $f=f\cdot j$\; berechnet, wobei $j = k - i$. Also ist $f = f \cdot (k-i)$. Da Invariante für $i$ korrekt ist, können wir $f$ ersetzen: $f = k!/(k-i)! \cdot (k-i) = k!/(k-(i+1))$. Das ist die Invariante für die $i+1$.\ Iteration. Damit ist die Schleifeninvariante auch weiterhin erfüllt.
      \item[Terminierung] Es gibt genau $k-1$ Iterationen, da $j$ in jeder Iteration dekrementiert wird. Wegen der Schleifeninvariante ist dann $f = k!/(k-(k-1))! = k!/1! = k!$.
    \end{description}
  \end{solution}
\end{parts}

\titledquestion{Kosinus rekursiv}
Es seien die folgenden Theoreme für $x,y\in\mathbb{R}$ gegeben:
\begin{description}
  \item[Additionstheorem:] $\cos (x + y) = 2 \cos x\cos y - \cos (x-y)$
  \item[Periodizität:] $\cos (x + 2\pi) = \cos x$
\end{description}

Wir betrachten im Folgenden nur positive Winkel. Die Kosinusfunktion lässt sich für kleine Winkel $x$ durch $\cos x\approx 1- x^2/2$ annähern. Diese Näherung gelte für $x < 0,001$. Die Operatoren $+,-,\cdot, /$ und $\operatorname{mod}$ werden in konstanter Zeit ausgewertet.

\begin{parts}
\part Geben Sie einen rekursiven Algorithmus zur Berechnung des Kosinus in Pseudocode an, der $\cos x$ für $x>0$ mithilfe des Additionstheorems berechnet. Die gegebene Näherung soll als Abbruchkriterium der Rekursionsgleichung dienen. Die Laufzeit des Algorithmus soll in $\Theta (\log x)$ liegen.
\begin{solution}
  Falls $x=y$, dann vereinfacht sich das Additionstheorem  zu $\cos (x + y) = 2 \cos x\cos y - 1$. Die Idee des rekursiven Algorithmus besteht darin, die Eingabe zu halbieren, sodass zwei halb so große Kosinuswerte berechnet werden müssen. Sobald der Eingabewert kleiner als $0,001$ ist, verwenden wir die gegebene Abschätzung für kleine Winkel als Abbruchbedingung.
  
  \begin{algorithm}[H]
    \If{$|x| < 0,001$}{
      \Return $1-x^2/2$
    }
    \ident{halfCos} = \ident{cos($x/2$)}\\
    \Return $2$ $ \cdot$ \ident{halfCos} $\cdot$ \ident{halfCos} $-$ $1$\\
    \caption{\ident{double cos($x$)}}
  \end{algorithm}
\end{solution}
\part Nun soll die Worst-Case-Laufzeit durch Periodizität verbessert werden. Wie kann dies geschehen? Welche Laufzeit ergibt sich dann?
  \begin{solution}
    Die Idee ist, den Eingabewert $x$ auf $2\pi$ zu beschränken. Dies können wir mit dem $\bmod$ Operator erreichen, der auf reellen Zahlen ein $c$ mit $x = a \cdot 2\pi  + c$ ausgibt. Da eine $a$ ganze Zahl sein muss, ist der Wert $c$ für ein bestimmtes $x$ eindeutig definiert. Zu Beginn des Algorithmus aus a) fügen wir die Zeile $x = x \bmod 2\pi$ ein. Es gilt dann $x < 2\pi$ und die Laufzeit ist nach oben beschränkt durch $\mathcal{O}(\log_2 2\pi)$. Dies ist ein konstanter Wert, sodass unsere Kosinusfunktion jetzt in konstanter Zeit läuft.
  \end{solution}
\part Was passiert, wenn Sie für $x < 0,001$ die Näherung $\cos x \approx 1$ verwenden?
\begin{solution}
  Die Rückgabe im Basisfall wäre dann immer $1$. Die Ebene über dem Basisfall, also wenn $0,001 \leq x < 0,002$  gilt, würde ebenfalls $1$ zurückgeben. Folglich würden alle Ebenen $1$ berechnen und die Funktion würde konstant $1$ zurückgeben. Dies ist offentsichtlich falsch, da der Kosinus keine konstante Funktion ist.
\end{solution}
\end{parts}

\titledquestion{Ähnliche Zahlen}
Sei \ident{A} ein Feld der Länge $n > 1$ von zufälligen Zahlen, wobei Zahlen mehrfach vorkommen dürfen.
\begin{parts}
\part Geben Sie einen Algorithmus in Pseudocode an, der zwei Zahlen \ident{A[$i$]} und \ident{A[$j$]} mit $i \neq j$ sucht, sodass~$\lvert$\ident{A[$i$]}$ - $\ident{A[$j$]}$\rvert$ minimal ist. Der Algorithmus soll die Indizes beider Zahlen ausgeben.
\begin{solution}

  \begin{algorithm}[H]
     $n$ = \ident{A.length}\\
     \ident{minI} = 1\\
     \ident{minJ} = 2\\
     \ident{min} = $\infty$\\
     \For{$i = 1$ \KwTo $n-1$}{
      \For{$j = i+1$ \KwTo $n$}{
        \ident{abs} = $|$\ident{A[$i$]} $-$ \ident{A[$j$]}$|$\\
        \If{\ident{abs} $<$ \ident{min}}{
          \ident{min} = \ident{abs}\\
          \ident{minI} = $i$\\
          \ident{minJ} = $j$\\
        }
      }
     }
     \Return (\ident{minI},\ident{minJ})
     \caption{\ident{findClosestPair(int[~] A)}}
  \end{algorithm}
\end{solution}
\part Begründen Sie die Korrektheit Ihres Algorithmus, indem Sie die Korrektheit der inneren Schleife mit einer Invariante zeigen.
\begin{solution}
  Wir geben eine Schleifeninvariante für die innere Schleife bei einem festen $i$ an: \enquote{Vor der $k$.\ Ausführung enthält \ident{min} den minimalen Abstand eines Tupels in der Menge \\ $\{(s,t) \in \mathbb{N}^2 \mid (s = i \Rightarrow t < i + k) \wedge (s < i \Rightarrow t < n + 1) \wedge (0 < s \leq i < t \leq n)\}$}.
  \begin{description}
    \item[Initialisierung] Vor der 1.\ Iteration der inneren \textbf{for}-Schleife kann \ident{min} nur den minimalen Abstand eines Tupels $(s,t)$  mit $s<i$ beinhalten, da \ident{min} noch gar nicht angefasst wurde.
    \item[Aufrechterhaltung] Gelte die Invariante vor der $k$.\ Iteration, also enthalte \ident{min} das Minimum obiger Menge. In der $k$.\ Iteration ist $j = i + k$. Falls nun $|$\ident{A[$i$]} $-$ \ident{A[$j$]}$|$ kleiner als das bisherige Minimum war, wird es ausgetauscht, andernfalls passiert nichts. Vor der $k+1$.\ Iteration ist also die Invariante wieder korrekt.
    \item[Terminierung] Bei Beendigung der \textbf{for}-Schleife sind $n - i$ Iterationen vergangen, also $k = n - i + 1$. Setzen wir diesen Wert in die obige Menge ein, fallen die ersten beiden Terme zusammen und \ident{min} enthält folglich das Minimum der Tupel in \\ $\{(s,t) \in \mathbb{N}^2 \mid  (t < n + 1) \wedge (0 < s \leq i < t \leq n)\}$.
  \end{description}
  Die Korrekheit der äußeren \textbf{for}-Schleife lässt sich analog zeigen; sie folgt sofort aus der Korrektheit der inneren \textbf{for}-Schleife.
\end{solution}
\end{parts}

\titledquestion{Noch mehr Korrektheitsbeweise und Rekursion}
Betrachten Sie den folgenden Algorithmus.

\begin{algorithm}[H]
  \KwData{Feld mit natürlichen Zahlen \ident{A}, natürliche Zahl $i$}
  \KwResult{Ein Wert, der mit \ident{A} zusammenhängt}
  \If{$i ==$ \ident{A.length}}{
    \Return  { \ident{A[$i$]}}
  }
  $k$ = \ident{doSomethingSimple}(\ident{A}, $i+1$);\\
  \If{$k > ${ \ident{A[$i$]}}}{
    \Return $k$
  }\Else{
    \Return{ \ident{A[$i$]}}
  }
 	\caption{\ident{int doSomethingSimple}(\textbf{int} \ident{A}[ ], \textbf{int} $i = 1$)}
\end{algorithm}
\begin{parts}
  \part Beschreiben Sie in einem Satz, was der Algorithmus macht.
  \begin{solution}
    Der Algorithmus findet das Maximum des Feldes \ident{A}.
  \end{solution}
  \part Beweisen Sie die Korrektheit des Algorithmus.
  \begin{solution}
    \begin{proof}
      Wir beweisen die Korrektheit per Induktion über $n= $ \ident{A.length}$ - i + 1$.
      \begin{description}
        \item[Induktionsanfang]  Sei $n=1$. Dann ist \ident{A.length} $ = i$ und die \textbf{If}-Abfrage in Zeile 1 wird mit \emph{wahr} ausgewertet. Es wird \ident{A[$i$]} zurückgegeben, was das Maximum des ein-elementigen Feldes \ident{A[A.length..A.length]} ist.
        \item[Induktionsschritt] Angenommen, der Algorithmus ist korrekt für $n$. Wir zeigen, dass er auch für $n+1$ korrekt ist. In Zeile 3 wird \ident{doSomethingSimple} für ein $i + 1$ aufgerufen, das heißt, für diesen Aufruf ist $n' = $ \ident{A.length}$ - (i+1) + 1 = $ \ident{A.length}$ - i < n$. Nach Induktionsannahme liefert Zeile 3 folglich das Maximum aus dem Teilfeld \ident{A[$i+1$..A-length]}. In der nun folgenden \textbf{If}-Abfrage wird das Maximum aus \ident{A[$i+1$..A-length]} und \ident{A[$i$]} ausgewählt. Damit liefert der Algorithmus insgesamt das Maximum aus \ident{A[$i$..A-length]}.
      \end{description}
    \end{proof}
  \end{solution}
  \part Geben Sie einen Algorithmus an, der äquivalent zu \ident{doSomethingSimple} ist, ohne Rekursion zu verwenden.
  \begin{solution}
  
    \begin{algorithm}[H]
      \If{\ident{A.length} $== 0$}{
        \Return $\emptyset$
      }
      \ident{max} = \ident{A[$1$]}\\
      \For{$i = 2 $ \KwTo \ident{ A.length}}{
        \lIf{\ident{A[$i$]} $>$ \ident{max}}{\ident{max} = \ident{A[$i$]}}
      }
      \caption{\ident{int findMax(int[~] A)}}
    \end{algorithm}
  \end{solution}
  \part Geben Sie eine Schleifeninvariante für Ihren inkrementellen Algorithmus an.
  \begin{solution}
    Vor der $j$.\ Iteration enthält \ident{max} immer das Maximum des Teilfelds \ident{A[1..$j$]}.
  \end{solution}
  \part Beweisen Sie die Korrektheit Ihres Algorithmus mit der von Ihnen aufgestellten Schleifeninvariante.
  \begin{solution}
    \begin{description}
      \item[Intialisierung] Unmittelbar vor der ersten ($j=1$) Iteration der \textbf{for}-Schleife steht in \ident{max} der Wert von \ident{A[$i$]}. Damit ist \ident{max} das Maximum aus \ident{A[1..$1$]}, und die Invariante ist erfüllt.
      \item[Aufrechterhaltung] Es gelte die Schleifeninvariante vor der $j$ Ausführung, das heißt, \ident{max} enthält das Maximum aus \ident{A[1..$j$]}. In der $j$. Iteration ist $i$=$j+1$ und der Wert von \ident{max} wird in der \textbf{if}-Abfrage mit \ident{A[$i$]} ausgetauscht, falls \ident{A[$i$]} größer als \ident{max} ist. Nach Auswertung von Zeile 5  enthält \ident{max} das Maximum aus  \ident{A[1..$i$]}. Da $j=i+1$, enthält \ident{max} das Maximum aus \ident{A[1..$j+1$]}, womit die Schleifeninvariante vor der $j+1$.\ Iteration erfüllt ist.
      \item[Terminierung] Die Schleife wird \ident{A.length}$-1$ mal durchlaufen. Damit gilt nach der letzten Iteration wegen der Schleifeninvariante, dass \ident{max} das Maximum aus \ident{A[1..A.length$-1+1$]} enthält.
    \end{description}
  \end{solution}
\end{parts}
 \titledquestion{$\mathcal{O}$-, $\Theta$- und $\Omega$-Notation}
Beweisen oder widerlegen Sie die Behauptungen. Arbeiten Sie mit der Definition aus der Vorlesung.
\begin{parts}
\part $f(n) = \frac{1}{2} n - 2 \in \Omega(\log_2 n)$
\begin{solution}
  Die Aussage ist \emph{wahr}. Dementsprechend ist zu zeigen: $\exists n_0 \exists c \forall n \geq n_0 \colon c \cdot \log_2 n \leq \frac{1}{2}n-2$
  Man wähle $c = 1/2$, dann gilt:
  \[\frac{1}{2}\log_2 n \leq  \frac{1}{2}n-2 \Leftrightarrow \log_2 n + 4 \leq n \]
  Also wähle $n_0 = 8$, da $\log_2 n \leq n$ für alle $n > 0$.
\end{solution}
\part $f(n) = n^n + n^2 \in O(n^{n-1})$
\begin{solution}
  Die Aussage ist \emph{falsch}. Dementsprechend ist zu zeigen: $\forall c \forall n \exists n_0 \geq n \colon n^n + n^2 > c \cdot n^{n-1}$
  Seien $n_0$ und $c$ beliebig.
  \begin{equation*}
    \begin{split}
      c \cdot n^{n-1} &< n^n + n^2 \\
      c \cdot n^{n-1} &< n^n < n^n + n^2\\
      c &< \frac{n^n}{n^{n-1}} < \frac{n^n}{n^{n-1}} \frac{n^n}{n^2}\\
      c &< n < n+\frac{n^n}{n^{n-2}}
    \end{split}
  \end{equation*}
  Damit wählen wir $n_0 = \max(n_0,c+1)$.
\end{solution}
\part $f(n) = \frac{n^4 - 4n^2}{2n + 7} \notin O(n^3)$
\begin{solution}
  Die Aussage ist \emph{falsch}. Dementsprechend ist zu zeigen: $\exists n_0 \exists c \forall n \geq n_0 \colon \frac{n^4 - 4n^2}{2n + 7} \leq c \cdot n^3$
  Wir vereinfachen die Ungleichung:
  \begin{equation*}
  \begin{split}
      \frac{n^4 - 4n^2}{2n + 7} &\leq c \cdot n^3\\
      n^4 - 4n^2 &\leq c \cdot n^3 \cdot (2n+7) \\
      n^4 - 4n^2 &\leq c \cdot (2n^4 + 7n^3) \\
  \end{split}
  \end{equation*}
  Mit der letzten Ungleichung wählen wir $n_0 = 1$ und $c = 1$.
\end{solution}
\part $f(n) = \log_3 (n^5 \cdot 9^{n^2}) \in \Omega(n \log_3 n)$
\begin{solution}
  Die Aussage ist \emph{wahr}. Also ist zu zeigen: $\exists n_0 \exists c \forall n \geq n_0 \colon c \cdot n \log_3 n \leq \log_3 (n^5 \cdot 9^{n^2})$.
  Wir zeigen dies:
  \begin{equation*}
    \begin{split}
      c \cdot n \log_3 n &\leq \log_3 (n^5 \cdot 9^{n^2}) = \log_3 n^5 + \log_3 9^{n^2} = 5\log_3 n + n^2 \log_3 9 = 5\log_3 n + 2n^2\\
      c \cdot n \log_3 n &\underbrace{\leq}_{\text{Wir müssen dieses zeigen}} 2n^2 \leq  5\log_3 n + 2n^2\\
      c \cdot \log_3 n&\leq 2n
    \end{split}
  \end{equation*}
      Also können wir $c=1$ wählen und $n_0 = 0$, da $\log_3 n \leq 2n$ für alle $n$.
\end{solution}
\part $f(n) = \log_a n \in \Theta (\log_b n)$ für beliebige $a,b>1$
\begin{solution}
  Die Aussage ist \emph{wahr}. Also ist zu zeigen: $\exists n_0 \exists c_1 \forall n \geq n_0 \colon c_1 \cdot \log_b n \leq \log_a n$ und $\exists n_0 \exists c_2 \forall n \geq n_0 \colon \log_a n \leq c_2 \cdot \log_b n$. Die Auflösung der Ungleichung liefert auch gleich die Werte für $c_1$ und $c_2$, unabhängig von $n$, also können wir $n_0 = 0$ wählen.
  \begin{equation*}
  \begin{split}
    c_1 \log_b n \leq & \log_a n\leq c_2 \log_b n \\
    c_1 \leq & \frac{\log_a n}{\log_b n} \leq c_2\\
    c_1 \leq & \frac{\log_b n}{\log_b a \cdot \log_b n} \leq c_2\\
    c_1 \leq & \frac{1}{\log_b a} \leq c_2
  \end{split}
  \end{equation*}
\end{solution}
\part $f(n) = \frac{1}{100} n^2 + n \sin n \in \Theta (n^2)$
\begin{solution}
  Die Aussage ist \emph{wahr}. Wir zeigen zunächst $f(n) \in \Omega(n^2)$, dafür ist folgendes zu zeigen: $\exists n_1 \exists c_1 \forall n \geq n_1 \colon c_1 \cdot n^2 \leq \frac{1}{100} n^2 + n \sin n$:
  \begin{equation*}
    \begin{split}
      c_1 \cdot n^2 &\leq \frac{1}{100} n^2 + n \sin n \\
      c_1 \cdot n^2 &\underbrace{\leq}_{\text{Das zeigen wir}} \frac{1}{100} n^2 - n \leq \frac{1}{100} n^2 + n \sin n\\
      c_1 \cdot n &\leq \frac{1}{100}n-1\\
    \end{split}
  \end{equation*}
  Nun wählen wir $n_1 = 101$, sodass $f(n) \geq 0$. Anschließend wählen wir passendes $c_1$, sodass die Ungleichung erfüllt ist, zum Beispiel $c_1 = 1/100 - 1/101$ (diesen Wert erhält man durch Multipliation obiger Ungleichung mit $1/n$. Jetzt zeigen wir noch $f(n) \in \mathcal{O}(n^2)$, dafür müssen wir zeigen, dass $\exists n_2 \exists c_2 \forall n \geq n_2 \colon \frac{1}{100} n^2 + n \sin n \leq c_2 \cdot n^2$:
  \begin{equation*}
    \begin{split}
      \frac{1}{100} n^2 + n \sin n &\leq c_2 \cdot n^2\\
      \frac{1}{100} n^2 + n \sin n \leq \frac{1}{100} n^2 + n &\underbrace{\leq}_{\text{Das zeigen wir}} c_2 \cdot n^2\\
      \frac{1}{100} n + 1 &\leq c_2 \cdot n \\
      \frac{1}{n} + \frac{1}{100} &\leq c_2
    \end{split}
  \end{equation*}
  Für $n_2 = 2$ und $c_2$ = $1$ ist die obige Ungleichung immer erfüllt. Für das geforderte $n_0$ nehmen wir $\max(n_1,n_2) = 100$.
\end{solution}
\part $f(n) = n^4 - 10 n^3 + 2 n \in O(n^3)$
\begin{solution}
  Die Aussage ist \emph{falsch}. Also ist zu zeigen: $\forall c \forall n \exists n_0 \geq n \colon n^4 - 10 n^3 + 2 n > c \cdot n^3$. Seien $c$ und $n_0$ beliebig:
  \begin{equation*}
  \begin{split}
     c \cdot n^3 &< n^4 - 10 n^3 + 2 n\\
     c \cdot n^3 &< n^4 - 10 n^3 < n^4 - 10 n^3 + 2 n\\
     c + 10 &< n
  \end{split}
  \end{equation*}
  Wähle also $n=\max(n_0, c + 11)$.
\end{solution}
\part $f(n) = \frac{9}{n} \notin \Omega(\frac{1}{\sqrt{n}})$
\begin{solution}
  Die Aussage ist \emph{korrekt}. Also ist zu zeigen: $\forall c \forall n \exists n_0 \geq n \colon c \cdot 1/\sqrt{n} > 9/n$. Seien $n_0$ und $c$ beliebig:
  \begin{equation*}
  \begin{split}
    \frac{9}{n} &< c \cdot \frac{1}{\sqrt{n}} \text{ Nächste Gleichung erhält man durch } n / \sqrt{n} = \sqrt{n}\\
        \frac{9}{c} &< \sqrt{n}\\
    \frac{81}{c^2} &< n \\
  \end{split}
  \end{equation*}
  Wähle also $n=\max(n_0, 81/c^2+1)$.
\end{solution}
\end{parts}
\titledquestion{Worst-Case-Laufzeiten für Algorithmen}
Wir betrachten folgendes Problem. Aus einem Eingabefeld \ident{A} von ganzen Zahlen sollen alle Tupel $(i,j)$ mit~$i < j$ ausgegeben werden, sodass \ident{A[$i$] + A[$j$]} ein Vielfaches von 10 ist. 
\begin{parts}
\part
Zeigen Sie, dass jeder Algorithmus, der dieses Problem löst, eine Worst-Case-Laufzeit von $\Omega(n)$ hat.
\begin{solution}
  Im Worst-Case ist jedes Element Teil einer Summe, die ein Vielfaches von 10 ist. Demnach müssen mindestens $n/2$ Tupel ausgegeben werden. Diese Laufzeit liegt in $\Omega(n)$. Eine Beispielliste ist \ident{A} $= [1, 9, 109, 91, 1099, 901, \dots]$.
\end{solution}
\part
Kann man eine asymptotisch größere (und damit bessere) untere Schranke angeben? Beweisen Sie Ihre Behauptung. Sie müssen dafür entweder zeigen, dass die Laufzeit für \emph{alle} Listen in $\Omega(n)$ liegt bzw. es \emph{eine} Familie von Listen der Länge $n$ gibt, für die die Laufzeit größer ist.
\begin{solution}
  Eine größere asymptotische Schranke für die Laufzeit ist $\Omega(n^2)$, wenn alle Zahlen in \ident{A} schon Vielfache von 10 sind. Dann sind alle \ident{A[$i$] + A[$j$]} für $i < j$ ebenfalls ein Vielfaches von 10. Die Anzahl der möglichen Summen bei $n$ Elementen ist $\sum_{i=1}^{n} i \in \Omega(n^2)$.
\end{solution}
\end{parts}

\titledquestion{Flugsicherheit}
Im Flugverkehr müssen die Flugzeuge gewisse Abstände einhalten. 
Gegeben ist eine unsortierte Liste von Flugzeugen.  Jedes Flugzeug $a$ hat drei Attribute, nämlich $a.x$, $a.y$ und $a.z$. Diese Attribute geben die Koordinaten im Luftraum an. Sie sollen einen Algorithmus angeben, der \ident{true} ausgibt, falls sich zwei Flugzeuge näher als den Abstand $d$ kommen.  Die Laufzeit Ihres Algorithmus soll $\mathcal{O}(n\log n)$ sein. Angenommen, Wurzelberechnungen sind zeitintensiv. Durch welche einfache Änderung muss der Algorithmus weniger Wurzelberechnungen durchführen?
\begin{solution}

  Der Abstand $e$ zwischen zwei Flugzeugen $f_1$ und $f_2$ ist mit der Formel 
  \[e = \sqrt{(f_1.x-f_2.x)^2+(f_1.y-f_2.y)^2+(f_1.z-f_2.z)^2}\]
 berechenbar. Da die Laufzeit im Bestcase in $\mathcal{O}(n\log n)$ soll, ist ein Vergleich jedes Flugzeugs mit jedem nicht möglich. Allerdings können die Flugzeuge nach einer Koordinate sortiert werden. Dann müssen nur diejenigen Flugzeuge miteinander verglichen werden, deren Abstand in dieser Koordinate kleiner als $d$ ist. Man kann nach jeder Koordinate sortieren, je nach Art des Flugraums ist eine Sortierung nach einer Koordinate sinnvoller als die andere. Beispielsweise kann die $x$-Koordinate aller Flugzeuge ähnlich sein, wenn sich die Flugzeuge im An- und Abflug einer Landebahn befinden. Auch die Höhe kann im freien Luftraum für viele Flugzeuge identisch sein. Für diesen Algorithmus wählen wir die $y$-Koordinate:
 
 \begin{algorithm}[H]
    \ident{mergeSort(planes)} \tcp{Sortiere nach $y$-Koordinate}
    \If{\ident{planes.length} $<$ $2$}{\Return \textbf{false}}
    \For{$i = 2$ \KwTo{}  \ident{planes.length}}{
      $f_1$ = \ident{planes[$i$]}\\
      $f_2$ = \ident{planes[$i-1$]}\\
      \ident{yDistance} = $|f_1.y - f_2.y|$\\
      \If{\ident{yDistance} $< d$}{
        $e = \sqrt{(f_1.x-f_2.x)^2+(f_1.y-f_2.y)^2+(f_1.z-f_2.z)^2}$\\
        \If{$e < d$}{
          \Return \textbf{true}
        }
      }
    }
    \Return \textbf{false}
    \caption{\ident{planesTooNear(Plane[~] planes, $d$)}}
 \end{algorithm}
\end{solution}
\titledquestion{Suppentöpfe}
Sie kennen das. Man will sich eine Nudelsuppe kochen, findet aber nicht den passenden Deckel für den Topf, da alle Deckel und Töpfe durcheinandergekommen sind. Da Sie immer auf Ihre Töpfe geachtet haben wissen Sie, dass zu jedem Topf ein Deckel vorhanden ist.
\begin{parts}
\part Sie möchten ein \emph{beliebiges} passendes Deckel-Topf-Paar finden. Wie viele Vergleiche sind dafür im besten Fall nötig?
\begin{solution}
Da ein \emph{beliebiges} Paar gesucht wird, nehmen Sie irgendeinen Topf und probieren alle Deckel aus. Im besten Fall passt gleich der erste Deckel, Sie benötigen nur einen Vergleich.
\end{solution}
\part Geben Sie einen Algorithmus in Pseudocode an, der ein Feld \ident{T} mit Topfgrößen und ein Feld mit Deckelgrößen \ident{D} entgegennimmt. Die Ausgabe soll aus zwei Indizes $i$ und $j$ bestehen, sodass \ident{D[$i$]} $=$ \ident{T[$j$]}. Wie viele Vergleiche braucht Ihr Algorithmus am schlechtesten Fall, um ein solches Paar zu finden? Können Sie Ihren Algorithmus verbessern, sodass er im schlechtesten Fall weniger Vergleiche braucht?
\begin{solution}
  Die Algorithmus soll also ein \emph{beliebiges} passendes Paar finden. Wir suchen einfach den Deckel zum ersten Topf:
  
  \begin{algorithm}[H]
    \For{$ j = 1$ \KwTo{} \ident{D.length}}{
      \If{\ident{D[$j$]} $=$ \ident{T[$1$]}}{
      \tcc{Da das Topfset vollständig ist, wird die folgende Zeile irgendwann erreicht und der Algo gibt immer ein Ergebnis zurück.}
        \Return $(1,j)$ 
      }
    }
    \caption{\ident{findPair(int[~] T, int[~] D)}}
  \end{algorithm}
  Der Algorithmus braucht im schlechtesten Fall \ident{D.length} $ - 1$ Vergleiche. Dies kann nicht verbessert werden, da wir im schlechtesten Fall immer den passenden Topf zuletzt erwischen, egal welchen Topf wir zuerst auswählen.
\end{solution}
\part\label{exerc:topf1} Nun haben Sie genug von der Unordnung und möchten zu jedem Topf den passenden Deckel finden. Wie gehen Sie vor, um jedem Topf einen passenden Deckel zuzuordnen? Sie dürfen dabei nur Topf mit Topf und Deckel mit Deckel vergleichen. Verwenden Sie $\Theta(n \log n)$ Vergleiche.
\begin{solution}
  Man sortiert die Deckel und die Töpfe zum Beispiel mit Quicksort. Anschließend kann man den ersten Deckel auf den ersten Topf setzen \dots
\end{solution}
\part Lösen Sie nun Teilaufgabe c), aber diesmal sollen nur Vergleiche zwischen je einem Topf und einem Deckel verwendet werden. Die Anzahl der Vergleiche soll wieder in $\Theta(n \log n)$ liegen. Welchem Verfahren aus der Vorlesung ähnelt Ihre Vorgehensweise?
\begin{solution}
  Sie wählen einen beliebigen Deckel und sortieren alle Töpfe, die zu klein für den Deckel sind, nach links und alle Töpfe die zu groß für den Deckel sind, nach rechts. Es bleibt ein Topf in der Mitte übrig, der zu diesem Deckel passt. Nun nehmen Sie einen Topf aus der linken Topfreihe und sortieren die Deckel mit diesem. Deckel, die zu klein sind, kommen nach links, Deckel die zu groß sind, kommen nach rechts. Es bleibt wieder ein Deckel übrig, mit dem Sie ein Paar bilden können. Sie haben nun je eine linke und rechte Seite für Deckel und Töpfe. Sie wissen, dass die Deckel auf der linken Seite zu den Töpfen auf der linken Seite passen müssen und ebenso auf der rechten Seite. Sie wiederholen den Vorgang also für links und rechts rekursiv, bis Sie alle Paare gefunden haben. Dieses Vorgehen ähnelt der QuickSort-Sortierung.
\end{solution}
\end{parts}

\end{questions}
\end{document}
