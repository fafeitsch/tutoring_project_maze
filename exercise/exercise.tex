\documentclass{exam}
\usepackage[utf8x]{inputenc}
\usepackage[ngerman]{babel}
\usepackage[german,ruled,vlined,linesnumbered,commentsnumbered,algoruled]{algorithm2e}
\usepackage{amsmath}
\usepackage{amsfonts}
\usepackage{amssymb}
\usepackage{amsthm}
\usepackage{marvosym}
\usepackage{tikz}
\usepackage[pdftex,
            pdfauthor={Fabian Feitsch},
            pdfsubject={Irrgarten-Aufgabe},
            pdfproducer={Latex},
            pdfcreator={pdflatex}]{hyperref}
\usepackage{csquotes}
\usepackage{censor}
\usepackage{listings}
\usepackage{parskip}
\SetKw{To}{to}
\SetKw{KwInt}{int}
\SetKw{KwTrue}{true}
\SetKw{KwFalse}{false}
\SetKw{KwOr}{or}

\DontPrintSemicolon
\newcommand{\A}{\sffamily{A}\normalfont}
\newcommand{\ident}[1]{{\normalfont\sffamily{#1}}}
\renewcommand\familydefault{\sfdefault}

\pagestyle{headandfoot}
\runningheadrule
\header{\today}{Irrgarten-Aufgabe für Java-Vorlesung}{}
\footer{}{}{Seite \thepage}
\DeclareUnicodeCharacter{9608}{\censor{B}}

\title{\vspace{-1cm}\sffamily{\textbf{Irrgarten-Aufgabe für Java-Vorlesung}}}
\date{}

 
\begin{document}
%\printanswers
\maketitle

\qformat{\textbf{\sffamily{Aufgabe \thequestion: \thequestiontitle}} \hfill }
\renewcommand{\solutiontitle}{\noindent\textbf{Lösung: }}
\renewcommand{\questionshook}{%
\setlength{\leftmargin}{0pt}%
\setlength{\labelwidth}{-\labelsep}%
}
\noindent{}In dieser Aufgabe sollen Sie Schritt für Schritt ein Java-Programm entwickeln, welches den kürzesten Weg aus einem Irrgarten findet. Das Java-Programm soll später mit folgendem Befehl gestartet werden können:


\begin{lstlisting}
java de.yourname.maze.cmd.Solver /home/user/maze.txt 0 3 9 6
\end{lstlisting}

Das erste Argument ist der Dateipfad zu einer Datei, in der der Irrgarten definiert wird. Das zweite Argument gibt an, in welcher Zeile des Irrgartens gestartet werden soll. Das dritte Argument gibt die Start-Spalte an. Das vierte und fünfte Argument definieren den Ausgang des Irrgartens analog zur Start-Definition. Die Ausgabe des Programms soll den Irrgarten zeigen, sowie den kürzesten Weg durch den Irrgarten:

\begin{center}
\begin{minipage}{2cm}
\begin{verbatim}
███*██████
█***██   █
█*█    █ █
█*██████ █
█*█      █
█*█ █ ████
█*█ █    █
█*█ ████ █
█******█ █
██████*███
\end{verbatim}
\end{minipage}
\end{center}
\paragraph{Hinweise:} Achten Sie auf die Zugriffsmodifizierer der Klassen und Methoden. Sie dürfen keine öffentlichen Methoden außer denen programmieren, die explizit in derAufgabe verlangt werden. Sofern nicht anders angegeben, sind die Methoden nicht statisch. Sie dürfen jedoch beliebige \verb|private|-Methoden implementieren, wenn diese Ihnen helfen. Achten Sie auch darauf, für Ihre Klassen das in den jeweiligen Aufgaben verlangte Paket auszuwählen.

\begin{questions}
\titledquestion{Importieren des Irrgartens}\label{exec:importer}
Schreiben Sie eine Klasse \verb|Importer|, die im Paket \verb|de.yourname.maze| liegt. Diese Klasse soll eine öffentliche Methode \verb|readFromScanner| besitzen, die ein zweidimensionales \verb|char|-Feld zurückgibt (\verb|char[][]|). Diese Methode nimmt ein Object vom Typ \verb|Scanner| entgegen und soll vom Scanner Eingaben des folgenden Formates einlesen:

\begin{lstlisting}
10
Lorem ipsum dolor sit amet,
consectetur adipiscing elit,
sed do eiusmod tempor incididunt
ut labore et dolore magna aliqua.
Mauris ultrices eros in cursus
turpis massa. Tristique
sollicitudin nibh sit amet commodo.
At imperdiet dui accumsan sit amet.
Fringilla urna porttitor rhoncus dolor purus.
Posuere morbi leo urna molestie at
\end{lstlisting}

In der ersten Zeile steht stets eine Zahl, die angibt, wie viele Zeilen noch kommen. Die restlichen Zeilen stellen den eigentlichen Inhalt dar und sollen in ein \verb|char[][]|-Feld gepackt werden. Zum Beispiel soll die Methode mit obiger Eingabe ein \verb|char[][]| zurückgeben, sodass \verb|char[2][0] = 's'| und \verb|char[2][1] = 'e'|. Sie dürfen davon ausgehen, dass in der ersten Zeile immer eine Zahl steht und diese die richtige Anzahl an Zeilen angibt.

\paragraph{Tipp:} Lassen sich sich vom Scanner eine komplette Zeile als String \verb|line| auslesen und wandeln Sie diese mittels der Methode \verb|line.toCharArray()| in ein Feld vom Typ \verb|char[]| um.

\titledquestion{PrettyPrinter}

Schreiben Sie eine Klasse \verb|PrettyPrinter| im Paket \verb|de.yourname.maze|. Die Klasse soll ein privates, finales Attribut vom Typ \verb|char[][]| besitzen, sowie einen Konstruktor mit einem Parameter, der das private Attribut setzt. Weiterhin soll die Klasse eine Methode \verb|printPretty| besitzen, die einen \verb|PrintStream| entgegen nimmt und das einzige Klassenattribut vom Typ \verb|char[][]| so ausgibt wie die Eingabe aus Aufgabe~\ref{exec:importer} aussah, aber ohne die Zahl in der ersten Zeile.

\paragraph{Tipp:} Mittels \verb|new String(char[])| können Sie ein \verb|char[]| in einen String umwandeln. Sie benötigen den Tipp nicht unbedingt, aber er macht den Code kürzer.

\titledquestion{Die Domain-Klassen}
Im folgenden werden die Klassen definiert, die den Irrgarten später darstellen sollen. Es handelt sich um die drei Klassen \verb|Cell|, \verb|CellList| und \verb|CellListItem|. Legen Sie alle drei Klassen im Paket \verb|de.yourname.maze.domain| an.
\begin{parts}
  \part
  Wir definieren den Begriff \emph{Zelle} als ein freies Feld in unserem Irrgarten. Eine Zelle besitzt vier Eigenschaften: \verb|row|, \verb|col|, \verb|neighbours| und \verb|predecessor|. Die ersten beiden Attribute beschreiben den Ort, an dem sich die Zelle im Irrgarten befindet. Welchen Datentyp schlagen Sie vor, um \verb|row| und \verb|col| zu verwalten?
  \part
  Erstellen Sie eine Klasse \verb|Cell| mit den Attributen \verb|row| und \verb|col| sowie einen Konstruktor, mit dem beide Werte gesetzt werden können. Schreiben Sie außerdem Getter für beide Attribute. Legen Sie eine \verb|toString()|-Methode an, die beide Werte schön formatiert ausgibt.
  \part
  Das Attribut \verb|neighbours| sei vom Typ \verb|CellList| und das Attribut \verb|predecessor| sei vom Typ \verb|Cell|.
\end{parts}

\end{questions}
\end{document}
