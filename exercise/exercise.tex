\documentclass{exam}
\usepackage[utf8x]{inputenc}
\usepackage[ngerman]{babel}
\usepackage{amsmath}
\usepackage{amsfonts}
\usepackage{amssymb}
\usepackage{amsthm}
\usepackage{marvosym}
\usepackage{xcolor}
\usepackage[pdftex,
            pdfauthor={Fabian Feitsch},
            pdfsubject={Irrgarten-Aufgabe},
            pdfproducer={Latex},
            pdfcreator={pdflatex}]{hyperref}
\usepackage{csquotes}
\usepackage{censor}
\usepackage{listings}
\usepackage{parskip}

\definecolor{mygreen}{rgb}{0,0.6,0}
\definecolor{mygray}{rgb}{0.5,0.5,0.5}
\definecolor{mymauve}{rgb}{0.58,0,0.82}

\lstset{ %
  backgroundcolor=\color{white},   % choose the background color
  basicstyle=\ttfamily,
  breaklines=true,                 % automatic line breaking only at whitespace
  captionpos=b,                    % sets the caption-position to bottom
  commentstyle=\color{mygreen},    % comment style
  escapeinside={\%*}{*)},          % if you want to add LaTeX within your code
  keywordstyle=\color{blue},       % keyword style
  stringstyle=\color{mymauve},     % string literal style
}

\renewcommand\familydefault{\sfdefault}

\pagestyle{headandfoot}
\runningheadrule
\header{\today}{Irrgarten-Aufgabe für Java-Vorlesung}{}
\footer{}{}{Seite \thepage}
\DeclareUnicodeCharacter{9608}{\censor{B}}

\title{\vspace{-1cm}\sffamily{\textbf{Irrgarten-Aufgabe für Java-Vorlesung}}}
\date{}

 
\begin{document}
%\printanswers
\maketitle

\qformat{\textbf{\sffamily{Aufgabe \thequestion: \thequestiontitle}} \hfill }
\renewcommand{\solutiontitle}{\noindent\textbf{Lösung: }}
\renewcommand{\questionshook}{%
\setlength{\leftmargin}{0pt}%
\setlength{\labelwidth}{-\labelsep}%
}
\noindent{}In dieser Aufgabe sollen Sie Schritt für Schritt ein Java-Programm entwickeln, welches den kürzesten Weg aus einem Irrgarten findet. Das Java-Programm soll später mit folgendem Befehl gestartet werden können:


\begin{lstlisting}
java de.yourname.maze.cmd.Solver /home/user/maze.txt 0 3 9 6
\end{lstlisting}

Das erste Argument ist der Dateipfad zu einer Datei, in der der Irrgarten definiert wird. Das zweite Argument gibt an, in welcher Zeile des Irrgartens gestartet werden soll. Das dritte Argument gibt die Start-Spalte an. Das vierte und fünfte Argument definieren den Ausgang des Irrgartens analog zur Start-Definition. Die Ausgabe des Programms soll den Irrgarten zeigen, sowie den kürzesten Weg durch den Irrgarten:

\begin{center}
\begin{minipage}{2cm}
\begin{verbatim}
███*██████
█***██   █
█*█    █ █
█*██████ █
█*█      █
█*█ █ ████
█*█ █    █
█*█ ████ █
█******█ █
██████*███
\end{verbatim}
\end{minipage}
\end{center}
\paragraph{Hinweise:} Achten Sie auf die Zugriffsmodifizierer der Klassen und Methoden. Sie dürfen keine öffentlichen Methoden außer denen programmieren, die explizit in derAufgabe verlangt werden. Sofern nicht anders angegeben, sind die Methoden nicht statisch. Sie dürfen jedoch beliebige \verb|private|-Methoden implementieren, wenn diese Ihnen helfen. Achten Sie auch darauf, für Ihre Klassen das in den jeweiligen Aufgaben verlangte Paket auszuwählen.

\begin{questions}
\titledquestion{Importieren des Irrgartens}\label{exec:importer}
Schreiben Sie eine Klasse \verb|Importer|, die im Paket \verb|de.yourname.maze| liegt. Diese Klasse soll eine öffentliche Methode \verb|readFromScanner| besitzen, die ein zweidimensionales \verb|char|-Feld zurückgibt (\verb|char[][]|). Diese Methode nimmt ein Object vom Typ \verb|Scanner| entgegen und soll vom Scanner Eingaben des folgenden Formates einlesen:

\begin{lstlisting}
10
Lorem ipsum dolor sit amet,
consectetur adipiscing elit,
sed do eiusmod tempor incididunt
ut labore et dolore magna aliqua.
Mauris ultrices eros in cursus
turpis massa. Tristique
sollicitudin nibh sit amet commodo.
At imperdiet dui accumsan sit amet.
Fringilla urna porttitor rhoncus dolor purus.
Posuere morbi leo urna molestie at
\end{lstlisting}

In der ersten Zeile steht stets eine Zahl, die angibt, wie viele Zeilen noch kommen. Die restlichen Zeilen stellen den eigentlichen Inhalt dar und sollen in ein \verb|char[][]|-Feld gepackt werden. Zum Beispiel soll die Methode mit obiger Eingabe ein \verb|char[][]| zurückgeben, sodass \verb|char[2][0] = 's'| und \verb|char[2][1] = 'e'|. Sie dürfen davon ausgehen, dass in der ersten Zeile immer eine Zahl steht und diese die richtige Anzahl an Zeilen angibt.

\paragraph{Tipp:} Lassen sich sich vom Scanner eine komplette Zeile als String \verb|line| auslesen und wandeln Sie diese mittels der Methode \verb|line.toCharArray()| in ein Feld vom Typ \verb|char[]| um.

\titledquestion{PrettyPrinter}

Schreiben Sie eine Klasse \verb|PrettyPrinter| im Paket \verb|de.yourname.maze|. Die Klasse soll ein privates, finales Attribut vom Typ \verb|char[][]| besitzen, sowie einen Konstruktor mit einem Parameter, der das private Attribut setzt. Weiterhin soll die Klasse eine Methode \verb|printPretty| besitzen, die einen \verb|PrintStream| entgegen nimmt und das einzige Klassenattribut vom Typ \verb|char[][]| so ausgibt wie die Eingabe aus Aufgabe~\ref{exec:importer} aussah, aber ohne die Zahl in der ersten Zeile.

\textbf{Tipp:} Mittels \verb|new String(char[])| können Sie ein \verb|char[]| in einen String umwandeln. Sie benötigen den Tipp nicht unbedingt, aber er macht den Code kürzer.

\titledquestion{Die Domain-Klassen}\label{exec:domain}
Im folgenden werden die Klassen definiert, die später den Irrgarten darstellen sollen. Es handelt sich um die drei Klassen \verb|Cell|, \verb|CellList| und \verb|CellListItem|. Legen Sie alle drei Klassen im Paket \verb|de.yourname.maze.domain| an. Diese Klassen werden in den weiteren Aufgaben um zusätzliche Methoden ergänzt.

\begin{parts}
  \part
  Wir definieren den Begriff \emph{Zelle} als ein freies Feld in unserem Irrgarten. Eine Zelle besitzt vier Eigenschaften: \verb|row|, \verb|col|, \verb|neighbours| und \verb|predecessor|. Die ersten beiden Attribute beschreiben den Ort, an dem sich die Zelle im Irrgarten befindet. Welchen Datentyp schlagen Sie vor, um \verb|row| und \verb|col| zu verwalten?
  \part
  Erstellen Sie eine Klasse \verb|Cell| mit den Attributen \verb|row| und \verb|col| sowie einen Konstruktor, mit dem beide Werte gesetzt werden können. Schreiben Sie außerdem Getter für beide Attribute. Legen Sie eine \verb|toString()|-Methode an, die den Namen der Klasse sowie beide Werte schön formatiert ausgibt.
  \part 
  Die Nachbarn einer Zelle $z$ seien freie Zellen, die direkt neben $z$ liegen. Eine Zelle kann somit maximal vier Nachbarn haben (oder weniger). Wir programmieren nun einen Datentyp \verb|CellList|, der eine List von Zellen darstellt. Legen Sie hierfür zunächst eine Klasse \verb|CellListItem| an, die ein Attribut \verb|content| vom Typ \verb|Cell| und ein Attribut \verb|next| vom Typ \verb|CellListItem| besitzt. Schreiben Sie einen Konstruktor, der das \verb|content|-Attribut setzt, sowie Getter/Setter für beide Attribute.
  \part 
  Schreiben Sie eine Klasse \verb|CellList| die genau ein Feld \verb|first| vom Typ \verb|CellListItem| besitzt. Legen Sie keinen Konstruktor an.  
  \part
  Erweitern Sie nun die Klasse \verb|Cell| um das Attribut \verb|neighbours| vom Typ \verb|CellList| und das Attribut \verb|predecessor| vom Typ \verb|Cell|.
\end{parts}

\titledquestion{Implementieren der Listen-Funktionalität}

Bis jetzt hat die Klasse \verb|CellList| außer einem Attribut keine weitere Funktion. In dieser Aufgabe wird die eigentliche Liste implementiert. Es handelt sich um eine \emph{einfach verkettete} Liste, das heißt, jedes \verb|CellListItem| zeigt auf seinen Nachfolger. Falls dieser nicht existiert, so ist der Nachfolger \verb|null|. Analog hat das Attribut \verb|first| der Klasse \verb|CellList| den Wert \verb|null| wenn die List leer ist. Die folgenden Methoden sollen in der Kasse \verb|CellList| implementiert werden.

\begin{parts}
	\part Implementieren Sie die Methode \verb|addToFront(Cell cell)|. Diese Methode nimmt ein Argument vom Typ \verb|Cell| entgegen und fügt ein neues \verb|CellListItem| mit \verb|cell| als Inhalt vorne in die Liste ein.
	\part Implementieren Sie die Methode \verb|poll()|, welche das erste Element der Liste entfernt und dessen \verb|Cell|-Inhalt zurück gibt. Falls die List leer ist, so soll \verb|null| zurückgegeben werden.
	\part Implementieren Sie die Methode \verb|enqueue(Cell cell)|, welche die übergebene Zelle \emph{hinten} an die List anhängt (vergleiche Aufgabe a)).
	\part Implementieren Sie die Methode \verb|isEmpty()|, die \verb|true| zurückgibt, wenn die Liste leer ist, ansonsten \verb|false|.
	\part Implementieren Sie die Methode \verb|getSize()|. Sie soll die aktuelle Größe der Liste zurückgeben.
	\part Implementieren Sie die Methode \verb|toArray()|. Sie soll ein Array vom Typ \verb|Cell[]| zurückgeben, welches alle in der Liste enthaltenen Zellen enthält. Der Inhalt der Liste soll sich nicht ändern. \textbf{Tipp:} Verwenden Sie hierfür die Methode \verb|getSize()|.	
\end{parts}

\titledquestion{Erweiterung der Zelle}

In Aufgabe~\ref{exec:domain} wurde bereits die Methode \verb|Cell| angelegt, die unter anderem ein Attribut \verb|CellList| besitzt, um die Nachbarn der Zelle zu verwalten. Fügen Sie der Klase \verb|Cell| nun weitere Methoden hinzu:

\begin{itemize}
\item Die Methode \verb|addNeighbour(Cell cell)| soll in der Nachbarsliste vorne die Zelle \verb|cell| einfügen.
\item Implementieren Sie außerdem die Methode \verb|getNeighboursAsArray()| welche die Nachbarn der Zelle als \verb|Cell[]| zurückgeben soll.
\item Legen Sie außerdem Getter und Setter für das \verb|predecessor|-Attribut an.
\item Die Methode \verb|wasVisited()| soll \verb|true| zurückgeben, falls der \verb|successor| ungleich \verb|null| ist, ansonsten \verb|false|.
\end{itemize}

\titledquestion{Konvertierung des Irrgartens in Zellen}\label{exec:mazebuilder}

Der \verb|Importer| aus Aufgabe~\ref{exec:importer} liest bis jetzt nur ein \verb|char[][]|-Array ein, welches wir jetzt zu einem Irrgarten weiterverarbeiten wollen.

\begin{parts}
\part Legen Sie eine Klasse \verb|MazeBuilder| im Paket \verb|de.yourname.maze.domain|. Diese Klasse soll keine Attribute und einen privaten (!) Konstruktur ohne Argumente besitzen.
\part Legen Sie in der Klasse eine private, statische Methode \verb|setNeighbours(Cell[][] cells)| an, die sich wie folgt verhält:
\begin{enumerate}
	\item Es wird nacheinander jede \verb|Cell| aus \verb|cells| betrachtet. Nenne die gerade betrachtete Zelle \verb|cell|.
	\item Falls \verb|cell == null| überspringe \verb|cell| und mache mit der nächsten Zelle weiter.
	\item Überprüfe, ob die Zelle \verb|c| direkt über \verb|cell != null|, falls ja, dann \verb|cell.addNeighbour(c)|.
	\item Überprüfe, ob die Zelle \verb|c| direkt unter \verb|cell != null|, falls ja, dann \verb|cell.addNeighbour(c)|.
	\item Überprüfe, ob die Zelle \verb|c| direkt links neben \verb|cell != null|, falls ja, dann \verb|cell.addNeighbour(c)|.
	\item Überprüfe, ob die Zelle \verb|c| direkt rechts neben \verb|cell != null|, falls ja, dann \verb|cell.addNeighbour(c)|.
\end{enumerate}
\textbf{Hinweis:} Achten Sie auf die Array-Begrenzungen. Wenn Sie bespielsweise eine Zelle in der ersten Zeile betrachten, dann können Sie Schritt 3) nicht ausführen. Analog gilt das auch für die anderen Abfragen. Nutzen Sie geeignete \verb|if|-Abfragen, um diese Fälle abzufangen.
\part Legen Sie nun eine öffentliche, statische Methode \verb|buildMaze| mit dem Rückgabewert \verb|Cell|. Die Methode hat drei Argumente, nämlich \verb|char[][] maze, int startRow| und \verb|int startCol|. Implementieren Sie die folgenden Schritte:
\begin{enumerate}
	\item Erstelle ein Feld \verb|cells| vom Typ \verb|Cell[][]|.
	\item Sorgen Sie dafür, dass \verb|cells| die gleichen Dimensionen wie \verb|maze| hat. Zum Beispiel soll aus
	\begin{verbatim}
		maze = [['a','b','c'],['d','e','f','g'],['h','j']]
	\end{verbatim}
	dieses Array vom Typ \verb|Cell[][]| werden:
	\begin{verbatim}
		cells = [[null,null,null],[null,null,null,null],[null,null]].
	\end{verbatim}
	Die Länge der Zeilen und Spalten ist also identisch zu \verb|maze|, aber alle Einträge sind \verb|null|.
	\item Setze nun \verb|cell[row][col] = new Cell(row, col)| falls \verb|maze[row][col] == ' '|. Die Werte der beiden Variablen \verb|row| und \verb|col| sollen über eine verschachtelte \verb|for|-Schleife erzeugt werden.
	\item Rufe \verb|setNeighbours(cells)| auf.
	\item Gebe \verb|cells[startRow][startCol]| zurück.
\end{enumerate}
\end{parts}

\titledquestion{Finden des kürzesten Weges}

Legen Sie eine Klasse \verb|ShortestPathFinder| im Paket \verb|de.yourname.maze.domain| mit einem Attribut \verb|char[][] maze]|, welches über den Konstruktur gesetzt wird.

Um den kürzesten Weg durch den Irrgarten zu finden wird der \emph{Breitensuche}-Algorithmus verwendet. Dieser beginnt mit einer Start-Zelle und erforscht den Irrgarten, bis die Ziel-Zelle gefunden wurde. Während des Algorithmuses wird der kürzeste Weg über die \verb|predecessor|-Attribute der \verb|Cell|-Klasse gespeichert. Am Ende kann, startend von der Ziel-Zelle, der Weg über die \verb|predecessor|-Attribute rekonstruiert werden. Dieser Weg ist dann der kürzeste Weg durch den Irrgarten. Im folgenden finden Sie den Breitensuche-Algorithmus angegeben. Vervollständigen Sie die Lücken und fügen Sie die Methode dann in die Klasse \verb|ShortestPathFinder| ein:

\begin{lstlisting}[language=java]
private CellList findShortestPath(Cell start, int endRow, int endCol){
    CellList queue = new CellList();
    queue.enqueue(_____);
    Cell current = null;
    _____ (!queue.isEmpty()) {
    	current = queue.poll();
        if (current.getRow() == ______ && current.getCol() == ______) {
        	break;
        }
        ______ neighbours = current.getNeighboursAsArray();
        for (int i = 0; _____________________; i++) {
        	Cell neighbour = neighbours[i];
            if (neighbour.wasVisited() || neighbour == start) {
            	continue;
            }
            neighbour.setPredecessor(current);
            queue.enqueue(neighbour);
        }
    }

    CellList path = new CellList();
    path.__________(current);
    while (current.getPredecessor() != null) {
    	current = current.getPredecessor();
        path.addToFront(_______);
    }
    return ____;
}
\end{lstlisting}

\titledquestion{Markieren des kürzesten Weges}

Implementieren Sie nun eine öffentliche Methode \verb|drawShortestPathInMaze| in der Klasse \verb|ShortestPathFinder| mit den Parametern \verb|startRow|, \verb|startCol|, \verb|endRow| und \verb|endCol|. Diese Methode soll zunächst aus dem klasseneigenen Attribute \verb|maze| einen Irrgarten mit Hilfe der Klasse \verb|MazeBuilder| aus Aufgabe~\ref{exec:mazebuilder} bauen und dann den kürzesten Weg in dem gebauten Irrgarten ermitteln. Anschließend soll \verb|maze[row][col] = '*'| gesetzt werden, falls eine Zelle mit entsprechenden Koordinaten im berechneten kürzesten Weg enthalten ist.
\newpage
\titledquestion{Die Main-Methode}

Kopieren Sie die folgende Klasse in das Paket \verb|de.yourname.maze.cmd| und vervollständigen Sie die Lücken, sodass das Programm die Eingangs geschlilderte Funktionalität hat.

\begin{lstlisting}[language=java]
package de.yourname.maze.cmd;

import de.yourname.maze.________;
import de.yourname.maze._____________;
import de.yourname.maze.domain.ShortestPathFinder;

import java.io.FileInputStream;
import java.io.FileNotFoundException;
import java.util.Scanner;

public class Solver {
    //Ignore the throws clause, it is needed to read files.
    public static void main(String args[]) throws FileNotFoundException {
        if (args.length != _) {
            System.out.println("Give 5 arguments: [PATH_TO_FILE] [START_ROW] [START_COL] [END_ROW] [END_COL]");
        }
        int startX = _______._________________;
        int startY = _______._________________;
        int endX = _______._________________;
        int endY = _______._________args[4]_;

        Scanner scanner = new Scanner(new FileInputStream(args[0]));
        Importer importer = ___ __________;
        char[][] maze = ________._______________(scanner);
        scanner.close();

        ShortestPathFinder finder = new ShortestPathFinder(____);
        ______.__________________________________________________;

        PrettyPrinter printer = new PrettyPrinter(____);
        printer.___________(System.out);
    }
}
\end{lstlisting}

\end{questions}
\end{document}
